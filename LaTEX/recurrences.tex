\section{Relações de recorrência}
\

No capítulo anterior, o pseudocódigo para o algoritmo fatorial recursivo foi apresentado. Através dele, é observável que a solução para $a_r = r! \ (r \in \mathbb{N})$, tendo como \textbf{condição inicial} $a_0 = 1$, pode ser computada através do que é chamado de \textbf{relação de recorrência}, nesse caso $ra_{r-1}$. Em uma \textbf{função definida recursivamente}, dado uma condição inicial e uma descrição dos estágios subsequentes em função dos anteriores é possível avaliar estágios da função para um dado $n$. Se $f$ for definida recursivamente, então $f(n)$ é única para qualquer inteiro positivo $n$.

A \textbf{solução de uma relação de recorrência} é uma expressão que de o valor da função em termos de seus argumentos e que não dependa de sub expressões. Não existe um método geral para solucionar uma relação de recorrência arbitrária. Existem classes de relações de recorrência onde técnicas de solução são conhecidas.

Por vezes, encontrar a solução da recorrência é difícil e é possível contentar-se apenas com uma boa estimativa assintótica da cota superior.

\textbf{Exemplo 1}

Uma simples relação de recorrência, $a_n = a_{n-1} + 1$ e $a_0=0$ pode ter uma solução encontrada por substituição de alguns termos.
\[a_n = a_{n-1} + 1 \Rightarrow a_{n-1} = a_{n-2} + 1 \Rightarrow a_n = a_{n-2} + 2\]
\[a_n = a_{n-k} + k\]

Assumindo $k=n$ temos que $a_n=a_0 + n$ pois $n-k = 0$, como $a_0=0$, então $a_n=n$ é a solução da recorrência. Por último, a notação $T(n)=T(n-1)+1$ e $T(0)=0$ com solução $T(n) = n$ é frequentemente utilizada, é claro que $T(n) \in O(n)$.

{\raggedleft $\bigtriangleup$ \par}

\textbf{Exemplo 2}

Para encontrar a solução de $a_n = a_{n-1} + n + 1$ com $a_0 = 0$ também é possível reescrever $a_n$
\[a_n = a_{n-1} + n + 1 = (a_{n-2} + (n-1) + 1) + n + 1 = a_{n-2} + (n-1) + n + 2 = a_{n-3} + (n-2) + (n-1) + n + 3\]

\[a_n = a_{n-k} + (n-k+1) + ... + (n-1) + n + k\]

Supondo $k=n$

\[a_n = a_0 + 1 + ... + (n-1) + 2n = \frac{n(n+1)}{2} + n \in O(n^2).\]

{\raggedleft $\bigtriangleup$ \par}

\textbf{Exemplo 3}

Resolva a recorrência $a_n = a_{n/2} + n$ com $a_0 = 1$, note que faz sentido falar de $n$ apenas no conjunto de potências de $2$, logo $n=2^k$. Usando a notação $T(n)=T(2^k)$.

\[T(n) = T \Bigr(\frac{n}{2}\Bigr) + n = \Bigr[T\Bigr(\frac{n}{2^2}\Bigr) + \frac{n}{2}\Bigr] + n\]
\[T(n) = T\Bigr(\frac{n}{2^k}\Bigr) + \frac{n}{2^k} + ... + \frac{n}{2} + n\]

Fazendo $k=\log_2(n)$

\[T(n) = T(1) + n\sum_{i=1}^k \frac{1}{i} = 1 + 2n \in O(n).\]

{\raggedleft $\bigtriangleup$ \par}

\subsection{Teorema mestre}
\

Sejam $a\geq 1$ e $b>1$ constantes, $f(n)$ uma função e $T(n)$ definida em $\mathbb{N}_0$ pela recorrência $T(n) = aT\Bigr(\frac{n}{b}\Bigr) + f(n)$, $T(n)$ tem os seguintes limites assintóticos

\[f(n) = O\Bigr(\frac{n^{\log_ba}}{n^\epsilon}\Bigr), \ \epsilon>0 \Rightarrow T(n) = \Theta(n^{\log_ba}).\]

\[f(n) = \Theta(n^{\log_ba}) \Rightarrow T(n) = \Theta(n^{\log_ba} \log n).\]

\[f(n) = \Omega(n^{\log_ba}n^\epsilon), \ \epsilon>0, \ \exists c; \ af\Bigr(\frac{n}{b}\Bigr)\leq cf(n) \Rightarrow T(n) = \Theta(f(n)).\]


O Teorema mestre dá a solução assintótica para recorrências da forma $aT\Bigr(\frac{n}{b}\Bigr) + f(n)$. Muitas vezes saber a ordem de crescimento da recorrência já é suficiente e uma solução de fórmula fechada não é necessária.

De modo intuitivo, o teorema mestre que ao comparar $f(n)$ e $n^{\log_ba}$ o caso (1) ocorre se $f(n)$ for polinomialmente menor que $n^{\log_ba}$, no caso (2) se forem iguais e no caso (3) se $f(n)$ for polinomialmente maior que $n^{\log_ba}$. Como essa diferença por um fator polinomial $n^\epsilon$ é obrigatória existem classes de funções $f(n)$ que são menores que $n^{\log_ba}$ mas não polinomialmente menores, dessa forma, o primeiro caso do teorema mestre não se aplicaria, isso é análogo para o terceiro caso.


\textbf{Exemplo 1}

$T(n)=9T\Bigr(\frac{n}{3}\Bigr) + n$. Primeiramente, identificando os termos se obtém, $a=9$, $b=3$ e $f(n)=n$.

Em segundo lugar, avaliando $n^{\log_ba}=n^{\log_3 9} = n^2$. Tomando $\epsilon = 1$, $f(n) = O(n^{2-1})$ e $f(n) = n \prec n^2$, o primeiro caso é satisfeito e $T(n) = \Theta(n^2).$

{\raggedleft $\bigtriangleup$ \par}

\textbf{Exemplo 2}

$T(n) = 2T\Bigr(\frac{n}{4}\Bigr) + \sqrt{n}$. Para essa recorrência, $a = 2$, $b = 4$ e $f(n) = \sqrt{n}$, ao avaliar $n^{\log_ba}=n^{\log_4 2} = \sqrt{n}$. Como $f(n) = \Theta(\sqrt{n})$ o segundo caso é satisfeito e $T(n) = \Theta(\sqrt{n}\log n)$.

{\raggedleft $\bigtriangleup$ \par}

\textbf{Exemplo 3 (Teorema não se aplica)}

$T(n) = 6T\Bigr(\frac{n}{6}\Bigr) + {n \log n}$. Neste caso $n^{\log_ba} = n \prec n \log n$, entretanto, o terceiro caso não se aplica, observando que $f(n)$ não é polinomialmente maior que $n$, é apenas maior por um fator $\log n$.

{\raggedleft $\bigtriangleup$ \par}

\textbf{Exemplo 4}

$T(n) = 3T\Bigr(\frac{n}{4}\Bigr) + {n \log n}$. Identificando os termos na recorrência, $n^{\log_4 3} = n^{0,793}$. Para uma constante $\epsilon \approx 0,2$, $f(n) = \Omega(n^{0,793}n^{0,2})$. Verificando a condição de regularidade $af\Bigr(\frac{n}{b}\Bigr) = 3\Bigr(\frac{n}{4}\Bigr)\log\Bigr(\frac{3}{4}\Bigr) \leq \Bigr(\frac{3}{4}\Bigr) n \log n \leq \frac{3}{4}f(n)$ logo $c=\frac{3}{4}$. Assim, o terceiro caso do teorema mestre pode ser aplicado e $T(n) = \Theta (n \log n)$.

{\raggedleft $\bigtriangleup$ \par}