\section{Relações de recorrência}
\

No capítulo anterior, o pseudocódigo para o algoritmo fatorial recursivo foi apresentado. Através dele, é observável que a solução para $a_r = r! \ (r \in \mathbb{N})$, tendo como \textbf{condição inicial} $a_0 = 1$, pode ser computada através do que é chamado de \textbf{relação de recorrência}, nesse caso $ra_{r-1}$. Em uma \textbf{função definida recursivamente}, dado uma condição inicial e uma descrição dos estágios subsequentes em função dos anteriores é possível avaliar estágios da função para um dado $n$. Se $f$ for definida recursivamente, então $f(n)$ é única para qualquer inteiro positivo $n$.

A \textbf{solução de uma relação de recorrência} é uma expressão que de o valor da função em termos de seus argumentos e que não dependa de sub expressões. Não existe um método geral para solucionar uma relação de recorrência arbitrária. Existem classes de relações de recorrência onde técnicas de solução são conhecidas.

Por vezes, encontrar a solução da recorrência é difícil e é possível contentar-se apenas com uma boa estimativa assintótica da cota superior.

\subsection{Método da substituição}

\textbf{Exemplo 1}

Uma simples relação de recorrência, $a_n = a_{n-1} + 1$ e $a_0=0$ pode ter uma solução encontrada por substituição de alguns termos.
\[a_n = a_{n-1} + 1 \Rightarrow a_{n-1} = a_{n-2} + 1 \Rightarrow a_n = a_{n-2} + 2\]
\[a_n = a_{n-k} + k\]

Assumindo $k=n$ temos que $a_n=a_0 + n$ pois $n-k = 0$, como $a_0=0$, então $a_n=n$ é a solução da recorrência. Por último, a notação $T(n)=T(n-1)+1$ e $T(0)=0$ com solução $T(n) = n$ é frequentemente utilizada, é claro que $T(n) \in O(n)$.

{\raggedleft $\bigtriangleup$ \par}

\textbf{Exemplo 2}

Para encontrar a solução de $a_n = a_{n-1} + n + 1$ com $a_0 = 0$ também é possível reescrever $a_n$
\[a_n = a_{n-1} + n + 1 = (a_{n-2} + (n-1) + 1) + n + 1 = a_{n-2} + (n-1) + n + 2 = a_{n-3} + (n-2) + (n-1) + n + 3\]

\[a_n = a_{n-k} + (n-k+1) + ... + (n-1) + n + k\]

Supondo $k=n$

\[a_n = a_0 + 1 + ... + (n-1) + 2n = \frac{n(n+1)}{2} + n \in O(n^2).\]

{\raggedleft $\bigtriangleup$ \par}

\textbf{Exemplo 3}

Resolva a recorrência $a_n = a_{n/2} + n$ com $a_0 = 1$, note que faz sentido falar de $n$ apenas no conjunto de potências de $2$, logo $n=2^k$. Usando a notação $T(n)=T(2^k)$.

\[T(n) = T \Bigr(\frac{n}{2}\Bigr) + n = \Bigr[T\Bigr(\frac{n}{2^2}\Bigr) + \frac{n}{2}\Bigr] + n\]
\[T(n) = T\Bigr(\frac{n}{2^k}\Bigr) + \frac{n}{2^k} + ... + \frac{n}{2} + n\]

Fazendo $k=\log_2(n)$

\[T(n) = T(1) + n\sum_{i=1}^k \frac{1}{i} = 1 + 2n \in O(n).\]

{\raggedleft $\bigtriangleup$ \par}

\subsection{Teorema mestre}
\

Sejam $a\geq 1$ e $b>1$ constantes, $f(n)$ uma função e $T(n)$ definida em $\mathbb{N}_0$ pela recorrência $T(n) = aT\Bigr(\frac{n}{b}\Bigr) + f(n)$, $T(n)$ tem os seguintes limites assintóticos

\[f(n) = O\Bigr(\frac{n^{\log_ba}}{n^\epsilon}\Bigr), \ \epsilon>0 \Rightarrow T(n) = \Theta(n^{\log_ba}).\]

\[f(n) = \Theta(n^{\log_ba}) \Rightarrow T(n) = \Theta(n^{\log_ba} \log n).\]

\[f(n) = \Omega(n^{\log_ba}n^\epsilon), \ \epsilon>0, \ \exists c; \ af\Bigr(\frac{n}{b}\Bigr)\leq cf(n) \Rightarrow T(n) = \Theta(f(n)).\]


O Teorema mestre dá a solução assintótica para recorrências da forma $aT\Bigr(\frac{n}{b}\Bigr) + f(n)$. Muitas vezes saber a ordem de crescimento da recorrência já é suficiente e uma solução de fórmula fechada não é necessária.

De modo intuitivo, o teorema mestre que ao comparar $f(n)$ e $n^{\log_ba}$ o caso (1) ocorre se $f(n)$ for polinomialmente menor que $n^{\log_ba}$, no caso (2) se forem iguais e no caso (3) se $f(n)$ for polinomialmente maior que $n^{\log_ba}$. Como essa diferença por um fator polinomial $n^\epsilon$ é obrigatória existem classes de funções $f(n)$ que são menores que $n^{\log_ba}$ mas não polinomialmente menores, dessa forma, o primeiro caso do teorema mestre não se aplicaria, isso é análogo para o terceiro caso.

\textbf{Exemplo 1}

$T(n)=9T\Bigr(\frac{n}{3}\Bigr) + n$. Primeiramente, identificando os termos se obtém, $a=9$, $b=3$ e $f(n)=n$.

Em segundo lugar, avaliando $n^{\log_ba}=n^{\log_3 9} = n^2$. Tomando $\epsilon = 1$, $f(n) = O(n^{2-1})$ e $f(n) = n \prec n^2$, o primeiro caso é satisfeito e $T(n) = \Theta(n^2).$

{\raggedleft $\bigtriangleup$ \par}

\textbf{Exemplo 2}

$T(n) = 2T\Bigr(\frac{n}{4}\Bigr) + \sqrt{n}$. Para essa recorrência, $a = 2$, $b = 4$ e $f(n) = \sqrt{n}$, ao avaliar $n^{\log_ba}=n^{\log_4 2} = \sqrt{n}$. Como $f(n) = \Theta(\sqrt{n})$ o segundo caso é satisfeito e $T(n) = \Theta(\sqrt{n}\log n)$.

{\raggedleft $\bigtriangleup$ \par}

\textbf{Exemplo 3 (Teorema não se aplica)}

$T(n) = 6T\Bigr(\frac{n}{6}\Bigr) + {n \log n}$. Neste caso $n^{\log_ba} = n \prec n \log n$, entretanto, o terceiro caso não se aplica, observando que $f(n)$ não é polinomialmente maior que $n$, é apenas maior por um fator $\log n$.

{\raggedleft $\bigtriangleup$ \par}

\textbf{Exemplo 4}

$T(n) = 3T\Bigr(\frac{n}{4}\Bigr) + {n \log n}$. Identificando os termos na recorrência, $n^{\log_4 3} = n^{0,793}$. Para uma constante $\epsilon \approx 0,2$, $f(n) = \Omega(n^{0,793}n^{0,2})$. Verificando a condição de regularidade $af\Bigr(\frac{n}{b}\Bigr) = 3\Bigr(\frac{n}{4}\Bigr)\log\Bigr(\frac{3}{4}\Bigr) \leq \Bigr(\frac{3}{4}\Bigr) n \log n \leq \frac{3}{4}f(n)$ logo $c=\frac{3}{4}$. Assim, o terceiro caso do teorema mestre pode ser aplicado e $T(n) = \Theta (n \log n)$.

{\raggedleft $\bigtriangleup$ \par}

\subsection{Relações de recorrência homogêneas e não homogêneas}
\

O método da substituição pode ser difícil para algumas relações de recorrência e o método mestre é restrito para formatos de relações. Ainda existem classes que podem ser solucionadas de outras formas, similares a encontrar soluções para equações diferenciais simples.

Através dos métodos a seguir é possível solucionar recorrências como as somas de quadrados ou até mesmo a famosa sequência de Fibonacci.

\subsubsection{Princípio da superposição}
\

Se funções $g_i(n) (i = 1, ..., k)$ forem soluções para uma relação de recorrência linear com coeficientes constantes de ordem r, então a combinação linear das $k$ soluções, $A_1g_1(n) + ... + A_kg_k(n)$, é solução da relação de recorrência, $A_i \in \mathbb{R}$.

\subsubsection{Relações de recorrência homogêneas}
\

Uma \textbf{relação de recorrência linear com coeficientes constantes de ordem r} é definida como uma relação de recorrência da forma $a_n = k_1a_{n-1}+...+k_ra_{n-r} + f(n)$ sendo $k_i \ (i = 1, ... , r)$ constantes e $f(n) = 0$. Se $g(n)$ é uma função onde $a_n=g(n)$ então $g(n)$ é chamada de \textbf{solução} da relação de recorrência.

Digamos que $a_r = x^r$ é solução da relação, então $x^n = k_1x^{n-1}+...+k_rx^{n-r}$, e ignorando a solução trivial $x=0$, obtém-se a equação polinomial de grau $r$, $x^n - k_1x^{n-1}-...-k_rx^{n-r} = 0$, chamada de \textbf{equação característica} da relação de recorrência.

Se as raízes da equação características forem todas \textbf{reais e distintas}, a solução homogênea será da forma
\[a_n = A_1(x_1)^n + ... + A_n(x_n)^n.\]

\textbf{Exemplo 1}

$T(n) = 9T(n-2); \ T(0) = 6,  \ T(1) = 12$. Obtendo a equação característica, $x^2 - 9 = 0$, as raízes são avaliadas em $x_i = {-3,3}$. A solução da relação de recorrência é da forma $T(n) = A_1(-3)^n + A_2(3)^n$. Dadas as condições iniciais, $A_1 + A_2 = 6$ e $-3A_1 + 3A_2 = 12$, assim a solução é 
\[T(n) = (-3)^n + 5(3)^n.\]

\textbf{Exemplo 2 (Termo geral da sequência de Fibonacci)}

A sequência de Fibonacci com relação de recorrência $a_n = a_{n-1} + a_{n-2}$ com $a_0 = a_1 = 1$, pode ter uma solução obtida através do método apresentado.

As raízes da equação característica $x^2 - x - 1 = 0$ são $x_1 = \frac{1+\sqrt{5}}{2}$ e $x_2 = \frac{1 - \sqrt{5}}{2}$.

Usando a forma $a_n = A_1(x_1)^n + A_2(x_2)^n$, e as condições iniciais, avalia-se $A_1 = \frac{5 + \sqrt{5}}{10}$ e $A_2 = \frac{5 - \sqrt{5}}{10}$.

Dessa forma, multiplicando os termos, a solução para a relação de recorrência de Fibonacci é 

\[\frac{(\frac{1+\sqrt{5}}{2})^n-(\frac{1 - \sqrt{5}}{2})^n}{\sqrt{5}}.\]

{\raggedleft $\bigtriangleup$ \par}

Caso as raízes apresentarem \textbf{multiplicidade}, sendo reais, ou seja, \textbf{reais e repetidas}, o formato de solução para raízes reais e distintas não funcionará mais pois ele violaria o princípio da superposição da combinação linear.

Se $x$ for uma raiz de multiplicidade então a solução particular será da forma

\[u_p = (x)^n(A_1 + ... + A_{k}x^{k-1}).\]

\textbf{Exemplo 3}

Se $x_1 = 2$ tiver multiplicidade $3$ e $x_2 = 6$ multiplicidade $2$ e ambas forem raízes de uma mesma equação característica então o formato da solução será

\[a_n = 2^n(A_1 + A_2n + A_3n^2) + 6^n(B_1n + B_2n^2).\]

{\raggedleft $\bigtriangleup$ \par}

\subsubsection{Relações de recorrência não homogêneas}

Quando uma relação de recorrência é do formato $a_n = h_n + f(n)$ com $f(n) \neq 0$, dizemos que $h_n$ é a \textbf{parte homogênea} da relação de recorrência não homogênea. A solução geral será a soma da solução da parte homogênea com a solução não homogênea.

Existem dois casos especiais onde técnicas de solução são conhecidas. Primeiramente, o caso onde $f(n) = k(q)^n$, onde $q \neq 1$ é um número racional e $k$ uma constante conhecida. A escolha para uma solução particular é $u_n = A(q)^n$, a menos que a $q$ seja uma raiz da equação característica, nesse caso a escolha seria $u_n = A(n)^r(q)^n$, onde $r$ é a multiplicidade da raiz.

O segundo caso é quando $f(n) = k(n)^r$, se a equação característica não possuir raiz $x = 1$, a solução é do formato $A_1 + ... + A_k(n)^{k-1}$. Se $1$ for uma raiz de multiplicidade $r$ então a escolha será $A_1n^t + ... + A_kn^{t+(k-1)}$.

\textbf{Exemplo}

Avaliar a soma dos quadrados dos $n$ primeiros inteiros positivos.

A relação de recorrência é dada por $a_n = a_{n-1} + n^2$ com $a_0 = 0$. A solução homogênea é calculada como $x = 1$, de multiplicidade $1$, então a solução será da última forma apresentada.

A escolha para a solução particular não homogênea é $A_1n + A_2n^2 + A_3n^3$. Como $a_0 = 0$ a solução é da forma.

\[a_n = A_1n + A_2n^2 + A_3n^3,\]
\[a_n  = A_1(n-1) + A_2(n-1)^2 + A_3(n-1)^3 + n^2.\]

Colando os termos $n$ e $n^2$ em evidencia, um sistema pode ser formado para calcular $A_1 = \frac{1}{6}$, $A_2 = \frac{1}{2}$ e $A_3 = \frac{1}{3}$.

Dessa forma a solução é $a_n = \frac{n}{6} + \frac{n^2}{2} + \frac{n^3}{3}$, chegando no resultado

\[\sum_{i=0}^{n} i^2 = \frac{n(n+1)(2n+1)}{6}.\]

{\raggedleft $\bigtriangleup$ \par}

\subsection{Algoritmos recursivos}
\

Uma relação de recorrência é uma fórmula recursiva. Um procedimento recursivo na ciência da computação, chama a si próprio para resolver um problema similar porém menor, se a instância é pequena, ele a resolve diretamente como puder. Um exemplo de algoritmo recursivo é o de computar a soma dos números de Fibonacci até uma indexação $n$.

\begin{lstlisting}[language=C, frame=single]
    int fibonacci(n)
    {
        if(n <= 1) return n

        else return fibonacci(n-1) + fibonacci(n-2)
    }
\end{lstlisting}