\section{Teoria da Ordenação}
\

O \textbf{problema da ordenação} é recorrente na ciência da computação, o motivo para isso é bem simples: Muitos problemas algorítmicos são mais fáceis de serem resolvidos quando os dados estão distribuídos de forma ordenada.

Nas próximas sub sessões serão apresentados alguns algoritmos de ordenação, inicialmente serão apresentados algoritmos ingênuos com complexidade polinomial, logo após, algoritmos de ordenação eficientes e por fim algoritmos com complexidade de tempo de execução linear. Para fins de simplificação, os algoritmos serão apresentados para ordenar vetores de números inteiros, no entanto, qualquer tipo de dado pode ser ordenado por um parâmetro. 

\subsection{Ordenação por bolha}
\

O algoritmo de \textbf{ordenação por bolha}, \texttt{bubble sort}, também conhecido como ordenação por flutuação, é possivelmente um dos algoritmos de ordenação mais simples de serem entendidos, no entanto, ele apresenta uma complexidade de tempo de execução no pior caso que está em $O(n^2)$.

A ordenação por bolha compara um a um \textbf{pares} de elementos do vetor que são \textbf{adjacentes} colocando-os em suas devidas posições, e percorrendo o vetor várias vezes até que ele esteja ordenado.

Uma descrição das instruções do \texttt{bubble sort} pode ser dada como:

(1) Compare o par de elementos $k$ e $k+1$

\subsection{Ordenação por inserção}

\subsection{Ordenação por seleção}

\subsection{Merge sort}

\subsection{Shell sort}

\subsection{Quick sort}

\subsection{Counting sort}

\subsection{Bucket sort}

\subsection{Radix sort}

