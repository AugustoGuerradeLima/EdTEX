\section{Matrizes e vetores}

A álgebra linear desempenha um papel fundamental na matemática e na computação.

\subsection{Digressão em álgebra linear}

Sem o propósito de findar o tema, alguns conceitos de álgebra linear são relevantes para fundamentar a prática computacional dos algoritmos de matrizes e de operações vetoriais.

\subsubsection{Espaço vetorial}

Um espaço vetorial $V$ nada mais é que um conjunto, seus elementos são denominados vetores, no espaço vetorial existem operações de adição de dois vetores que resulta em uma soma, também pertencente ao espaço vetorial e multiplicação por uma escalar pertencente a um corpo (também um conjunto com propriedades definidas) por um vetor, que resulta em um produto escalar, pertencente ao espaço vetorial. 

Além disso, um espaço vetorial respeita axiomas como elemento neutro e comutatividade, que podem ser encontrados em livros de álgebra linear.

Exemplos de espaços vetoriais espaço, o conjunto ${0}$, o conjunto do números complexos $\mathbb{C}$ e o conjunto $F^{X,\mathbb{R}}$ de todas as funções $f_n: X \rightarrow \mathbb{R}$, onde $X$ é um conjunto não vazio.

\subsubsection{Subespaço vetorial}

É um subconjunto de um espaço vetorial, fechado por adição e multiplicação por escalar, além de possuir o elemento neutro $0$. Um conjunto $S$ é \textbf{conjunto gerador} de um espaço vetorial $V$ quando os elementos de $V$ podem ser expressos por \textbf{combinação linear} dos elementos de $S$.

Como exemplos, é possível dizer que qualquer reta que possa a origem é subespaço de $\mathbb{R}^n$; Que o subespaço formado por um vetor $v$ não nulo pertencente a $\mathbb{R}$ (espaço vetorial nos reais) é a reta que passa pela origem e contem $v$; E que para $u=[a,b]$ e $v=[c,d]$ serem múltiplos, é necessário e suficiente que $ad=bc$, então é possível dizer que se $v$ e $u\in \mathbb{R}^2$ não forem múltiplos, e forem vetores não nulos esses formarão um conjunto que será conjunto gerador de $\mathbb{R}^2$, argumentando que sua combinação linear é igual a um vetor contido no plano real, formando um sistema linear com determinante diferente de zero $(ad\neq bc \rightarrow ad - bc \neq 0)$, ou seja, possui solução para as constantes da combinação linear.

\subsection{Operações vetoriais}

\subsubsection{Norma euclidiana}

\begin{lstlisting}[language=C, frame=single]
float norma(float* v, int n)
{
    float somasq=0.0;
    for(int i=0;i<n;i++)
    {
        somasq+=v[i]*v[i];
    }
    return(sqrt(somasq));
}
\end{lstlisting}

\subsubsection{Produto escalar}

O produto escalar, ou produto interno, é uma operação entre dois vetores de tamanho $n$ que resulta em uma escalar. Definido por

\[u\cdot v = [a_0,...,a_k]\cdot[b_0,...,b_k] = a_0b_0+...+a_kb_k.\]

A definição geométrica do produto escalar, é o produto das normas euclidianas e do cosseno do ângulo entre esses vetores.

\[u\cdot v = |u||v|\cos(\theta).\]

Seu algoritmo em linguagem de programação C muito simples tem complexidade de tempo de execução $O(n)$, e complexidade de espaço $O(1)$.

\begin{lstlisting}[language=C, frame=single]
float produto_interno(float* u, float* v, int n)
{
    float w = 0.0;
    for(int i = 0; i<n; i++)
    {
        w+=u[i]*v[i];
    }
    return(w);
}
\end{lstlisting}

\subsubsection{Projeção}

A projeção do vetor $u$ sobre o vetor $v$, é uma operação que avalia a componente de $u$ que está na direção de $v$.

Definida por

\[proj_v(u) = \frac{u\cdot v}{|v|^2} v\]