\section*{Anexo I - Revisitando a linguagem de programação C}
\subsection*{Apontadores}
\

Para acessar o endereço de memória virtual de uma variável em C é utilizado o operador de referência $\&$, o endereço de memória é representado por 0x seguido de um número em base hexadecimal.

Um apontador é um tipo de variável cuja função é armazenar um endereço de memória virtual, se um apontador $p$ armazena o endereço de uma variável $x$ é possível então dizer que $p$ aponta para o endereço de $x$.

O processo de definir um apontador para um endereço significa que uma \textbf{referência} para o endereço é feita, já armazenar o identificador do endereço em uma variável é um processo de \textbf{dereferência}.

\begin{lstlisting}[language=C, frame=single]
    int x = 35;
    int* p = &x; //Referencia

    int d = *p; //Dereferencia
\end{lstlisting}

\subsection*{Aritmética de endereços}
\

A aritmética de endereços é importante para a alocação dinâmica. Na memória virtual, os endereços dos elementos de um vetor são armazenados de modo sequencial. Ao declarar um vetor $V$, sendo $i$ uma constante $i=0, 1 ...$ \textit{tamanho do vetor} $-1$, acessar $V[i]$ é idêntico a acessar $*(V+i)$.

\begin{lstlisting}[language=C, frame=single]
    int* V; 
    V = malloc (10*sizeof(int));
    //V[i]=*(V+i)
\end{lstlisting}

\subsection*{Alocação dinâmica}
\
